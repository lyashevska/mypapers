\documentclass{letter}
\usepackage{amssymb,amsmath}
\usepackage{graphicx}
\usepackage{geometry} 
\geometry{
 a4paper,
 total={210mm,297mm},
 left=20mm,
 right=20mm,
 top=20mm,
 bottom=20mm,
 }
\usepackage{lineno}
%\linenumbers
\usepackage{color, soul}
\newcommand{\xbf}{\ensuremath{\mathbf x}}

\begin{document}
\begin{letter}{
Prof. dr. ir. A. Stein\\
Editor in Chief\\
Spatial Statistics\\
        
\vspace{1cm}
Ms. Ref. No.:  \textbf{SPASTA-D-16-00008R1}\\
Title: `Grid-spacing and the quality of abundance maps for species that show spatial autocorrelation and zero-inflation'
}

\opening{Dear Professor Stein,}

We are grateful for the very useful review of our manuscript which you recently supplied. We have followed it to revise our work which we now offer for consideration as a revised manuscript. 

All typos and editing questions raised by the reviewers were addressed in the text.

Both reviwers suggested changes in colour coding for Figure 1 to make it more clear. We are grateful for these suggestions.
In particular, Reviewer 1 suggested to adjust  color coding for Figure 1 such that anything below 1 is a different grade of yellow. We have made these adjustments as we agree with the Reviewer 1 that it will facilitate comparison between the two maps.  
Reviewer 3, however, argued, that in black and white Figure 1 is difficult to understand. Here, Reviewer 3 probably means 'grayscale', as black and white is monochrome, i.e. has only two colours: black and white and cannot correctly reproduce images. This comment has already been raised (and adressed) in the previous version of revision. We created a palette (using bpy.colors(), package sp) with colors that differ in brightness/lightness as these are the only aspects that are conserved in grayscale. 

Reviewer 3: 

\textbf{comment}

Mostly if the authors want ecologists to use their method, more clarification about how to "translate" the values of the parameters into some guidelines would be needed (e.g., how much spatial autocorrelation is too much? How much zero-inflated is too much?).

\textbf{response}

How much spatial autocorrelation is too much?\\
We agree with the reviewer that this question is of great interest to ecologists, however we analysed only one species with one correlation strength, therefore we cannot make any comparison. An analysis of multiple species is not feasible due to the computing time involved in estimating the hyperparameters. However, we do believe that our study demonstrates a useful methodology that can be readily extended to any species that show similar distributional properties.

How much zero-inflation is too much?\\
Zero-inflation occurs when the incidence of zero counts is greater than expected for the Poisson distribution. 
There are statistical test for zero-inflation and overdispersion, but as far as we are aware there is no rule of thumb to decide when it is too much. Strictly speaking, once zero-inflation is detected, appropriate models should be used (e.g. Zero-Inflated Poisson or Negative Binomial). 

\textbf{comment} 

Page 3: "Bijleveld et al": Which year?

\textbf{response}

added

\textbf{comment} 

Page 3, Section 2.1 Data: This is a strange sampling design. What is the rational of such a sampling protocol? A figure would help to understand the sampling.

\textbf{response}

Designing this sampling protocol was not as a part of this paper, but it was designed by Bijleveld et al (2012). We refer to Figure 1 of their paper for more details. The grid was supplemented with the randomly chosen locations in order to calculate small-scale autocorrelation structure. 

\textbf{comment} 

Figure 2: Are these results based on 100 replicates? If so indicate it in the caption.

\textbf{response}

This is already explained in the last paragraph of section 2.3. All 100 grid-samples were used in prediction with the fixed parent-hyperparameters (blue), and 5 grid-samples were used in prediction with variable sample-specific estimates (red). We have now added this information to the caption.

\textbf{comment} 

Page 8, Line 177: "increase was from 0.006 at 200 m to 0.008 at 3200 m." For an ecologist's perspective what are these values correspond to? Please clarify how big (or not) difference it is.

\textbf{response}

Both values refer to the prevalence parameter $\pi$ of the Bernoulli distribution. An increase from 0.006 to 0.008 is not large, however the effect of grid spacing is more pronounced when hyperparameters are estimated from a sample. 

\textbf{comment} 

Also, please tell in a table how your parameters correspond to in terms of spatial autocorrelation and zero-inflected values. 

\textbf{response}

The reviewer probably refers to hyperparameters.  We agree that a table of the hyperparameter values used to create the pseudo-reality is useful. Such a table, however, is already presented as Table 1 in Lyashevska et al. (2016) which we closely followed in simulating fields with zero-inflated, spatially autocorrelated count data. Additionally, we have checked the hyperparameters values estimated from a sample (red stars on Figure 2) and found out that the range of the covariance function was alway bigger than a given grid spacing. For example, for the largest grid spacing of 3200 m the range of covariance function was 4760 m. 

\textbf{comment} 

Page 10, Section 4. Discussion and conclusions: Please comment on how your results are affected by the shape of the study area (i.e. a long and narrow rectangle): are there some potential edge effects?

\textbf{response}

We agree with the reviewer that this is an interesting question to answer. However the question of edge effects was not explicitly addressed in present study.  Potential edge effec was minimised at the sampling design phase done by Bijleveld et al (2012).

\end{letter}
\end{document}
